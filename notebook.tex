
% Default to the notebook output style

    


% Inherit from the specified cell style.




    
\documentclass[11pt]{article}

    
    
    \usepackage[T1]{fontenc}
    % Nicer default font (+ math font) than Computer Modern for most use cases
    \usepackage{mathpazo}

    % Basic figure setup, for now with no caption control since it's done
    % automatically by Pandoc (which extracts ![](path) syntax from Markdown).
    \usepackage{graphicx}
    % We will generate all images so they have a width \maxwidth. This means
    % that they will get their normal width if they fit onto the page, but
    % are scaled down if they would overflow the margins.
    \makeatletter
    \def\maxwidth{\ifdim\Gin@nat@width>\linewidth\linewidth
    \else\Gin@nat@width\fi}
    \makeatother
    \let\Oldincludegraphics\includegraphics
    % Set max figure width to be 80% of text width, for now hardcoded.
    \renewcommand{\includegraphics}[1]{\Oldincludegraphics[width=.8\maxwidth]{#1}}
    % Ensure that by default, figures have no caption (until we provide a
    % proper Figure object with a Caption API and a way to capture that
    % in the conversion process - todo).
    \usepackage{caption}
    \DeclareCaptionLabelFormat{nolabel}{}
    \captionsetup{labelformat=nolabel}

    \usepackage{adjustbox} % Used to constrain images to a maximum size 
    \usepackage{xcolor} % Allow colors to be defined
    \usepackage{enumerate} % Needed for markdown enumerations to work
    \usepackage{geometry} % Used to adjust the document margins
    \usepackage{amsmath} % Equations
    \usepackage{amssymb} % Equations
    \usepackage{textcomp} % defines textquotesingle
    % Hack from http://tex.stackexchange.com/a/47451/13684:
    \AtBeginDocument{%
        \def\PYZsq{\textquotesingle}% Upright quotes in Pygmentized code
    }
    \usepackage{upquote} % Upright quotes for verbatim code
    \usepackage{eurosym} % defines \euro
    \usepackage[mathletters]{ucs} % Extended unicode (utf-8) support
    \usepackage[utf8x]{inputenc} % Allow utf-8 characters in the tex document
    \usepackage{fancyvrb} % verbatim replacement that allows latex
    \usepackage{grffile} % extends the file name processing of package graphics 
                         % to support a larger range 
    % The hyperref package gives us a pdf with properly built
    % internal navigation ('pdf bookmarks' for the table of contents,
    % internal cross-reference links, web links for URLs, etc.)
    \usepackage{hyperref}
    \usepackage{longtable} % longtable support required by pandoc >1.10
    \usepackage{booktabs}  % table support for pandoc > 1.12.2
    \usepackage[inline]{enumitem} % IRkernel/repr support (it uses the enumerate* environment)
    \usepackage[normalem]{ulem} % ulem is needed to support strikethroughs (\sout)
                                % normalem makes italics be italics, not underlines
    

    
    
    % Colors for the hyperref package
    \definecolor{urlcolor}{rgb}{0,.145,.698}
    \definecolor{linkcolor}{rgb}{.71,0.21,0.01}
    \definecolor{citecolor}{rgb}{.12,.54,.11}

    % ANSI colors
    \definecolor{ansi-black}{HTML}{3E424D}
    \definecolor{ansi-black-intense}{HTML}{282C36}
    \definecolor{ansi-red}{HTML}{E75C58}
    \definecolor{ansi-red-intense}{HTML}{B22B31}
    \definecolor{ansi-green}{HTML}{00A250}
    \definecolor{ansi-green-intense}{HTML}{007427}
    \definecolor{ansi-yellow}{HTML}{DDB62B}
    \definecolor{ansi-yellow-intense}{HTML}{B27D12}
    \definecolor{ansi-blue}{HTML}{208FFB}
    \definecolor{ansi-blue-intense}{HTML}{0065CA}
    \definecolor{ansi-magenta}{HTML}{D160C4}
    \definecolor{ansi-magenta-intense}{HTML}{A03196}
    \definecolor{ansi-cyan}{HTML}{60C6C8}
    \definecolor{ansi-cyan-intense}{HTML}{258F8F}
    \definecolor{ansi-white}{HTML}{C5C1B4}
    \definecolor{ansi-white-intense}{HTML}{A1A6B2}

    % commands and environments needed by pandoc snippets
    % extracted from the output of `pandoc -s`
    \providecommand{\tightlist}{%
      \setlength{\itemsep}{0pt}\setlength{\parskip}{0pt}}
    \DefineVerbatimEnvironment{Highlighting}{Verbatim}{commandchars=\\\{\}}
    % Add ',fontsize=\small' for more characters per line
    \newenvironment{Shaded}{}{}
    \newcommand{\KeywordTok}[1]{\textcolor[rgb]{0.00,0.44,0.13}{\textbf{{#1}}}}
    \newcommand{\DataTypeTok}[1]{\textcolor[rgb]{0.56,0.13,0.00}{{#1}}}
    \newcommand{\DecValTok}[1]{\textcolor[rgb]{0.25,0.63,0.44}{{#1}}}
    \newcommand{\BaseNTok}[1]{\textcolor[rgb]{0.25,0.63,0.44}{{#1}}}
    \newcommand{\FloatTok}[1]{\textcolor[rgb]{0.25,0.63,0.44}{{#1}}}
    \newcommand{\CharTok}[1]{\textcolor[rgb]{0.25,0.44,0.63}{{#1}}}
    \newcommand{\StringTok}[1]{\textcolor[rgb]{0.25,0.44,0.63}{{#1}}}
    \newcommand{\CommentTok}[1]{\textcolor[rgb]{0.38,0.63,0.69}{\textit{{#1}}}}
    \newcommand{\OtherTok}[1]{\textcolor[rgb]{0.00,0.44,0.13}{{#1}}}
    \newcommand{\AlertTok}[1]{\textcolor[rgb]{1.00,0.00,0.00}{\textbf{{#1}}}}
    \newcommand{\FunctionTok}[1]{\textcolor[rgb]{0.02,0.16,0.49}{{#1}}}
    \newcommand{\RegionMarkerTok}[1]{{#1}}
    \newcommand{\ErrorTok}[1]{\textcolor[rgb]{1.00,0.00,0.00}{\textbf{{#1}}}}
    \newcommand{\NormalTok}[1]{{#1}}
    
    % Additional commands for more recent versions of Pandoc
    \newcommand{\ConstantTok}[1]{\textcolor[rgb]{0.53,0.00,0.00}{{#1}}}
    \newcommand{\SpecialCharTok}[1]{\textcolor[rgb]{0.25,0.44,0.63}{{#1}}}
    \newcommand{\VerbatimStringTok}[1]{\textcolor[rgb]{0.25,0.44,0.63}{{#1}}}
    \newcommand{\SpecialStringTok}[1]{\textcolor[rgb]{0.73,0.40,0.53}{{#1}}}
    \newcommand{\ImportTok}[1]{{#1}}
    \newcommand{\DocumentationTok}[1]{\textcolor[rgb]{0.73,0.13,0.13}{\textit{{#1}}}}
    \newcommand{\AnnotationTok}[1]{\textcolor[rgb]{0.38,0.63,0.69}{\textbf{\textit{{#1}}}}}
    \newcommand{\CommentVarTok}[1]{\textcolor[rgb]{0.38,0.63,0.69}{\textbf{\textit{{#1}}}}}
    \newcommand{\VariableTok}[1]{\textcolor[rgb]{0.10,0.09,0.49}{{#1}}}
    \newcommand{\ControlFlowTok}[1]{\textcolor[rgb]{0.00,0.44,0.13}{\textbf{{#1}}}}
    \newcommand{\OperatorTok}[1]{\textcolor[rgb]{0.40,0.40,0.40}{{#1}}}
    \newcommand{\BuiltInTok}[1]{{#1}}
    \newcommand{\ExtensionTok}[1]{{#1}}
    \newcommand{\PreprocessorTok}[1]{\textcolor[rgb]{0.74,0.48,0.00}{{#1}}}
    \newcommand{\AttributeTok}[1]{\textcolor[rgb]{0.49,0.56,0.16}{{#1}}}
    \newcommand{\InformationTok}[1]{\textcolor[rgb]{0.38,0.63,0.69}{\textbf{\textit{{#1}}}}}
    \newcommand{\WarningTok}[1]{\textcolor[rgb]{0.38,0.63,0.69}{\textbf{\textit{{#1}}}}}
    
    
    % Define a nice break command that doesn't care if a line doesn't already
    % exist.
    \def\br{\hspace*{\fill} \\* }
    % Math Jax compatability definitions
    \def\gt{>}
    \def\lt{<}
    % Document parameters
    \title{Kaplan-Meier}
    
    
    

    % Pygments definitions
    
\makeatletter
\def\PY@reset{\let\PY@it=\relax \let\PY@bf=\relax%
    \let\PY@ul=\relax \let\PY@tc=\relax%
    \let\PY@bc=\relax \let\PY@ff=\relax}
\def\PY@tok#1{\csname PY@tok@#1\endcsname}
\def\PY@toks#1+{\ifx\relax#1\empty\else%
    \PY@tok{#1}\expandafter\PY@toks\fi}
\def\PY@do#1{\PY@bc{\PY@tc{\PY@ul{%
    \PY@it{\PY@bf{\PY@ff{#1}}}}}}}
\def\PY#1#2{\PY@reset\PY@toks#1+\relax+\PY@do{#2}}

\expandafter\def\csname PY@tok@w\endcsname{\def\PY@tc##1{\textcolor[rgb]{0.73,0.73,0.73}{##1}}}
\expandafter\def\csname PY@tok@c\endcsname{\let\PY@it=\textit\def\PY@tc##1{\textcolor[rgb]{0.25,0.50,0.50}{##1}}}
\expandafter\def\csname PY@tok@cp\endcsname{\def\PY@tc##1{\textcolor[rgb]{0.74,0.48,0.00}{##1}}}
\expandafter\def\csname PY@tok@k\endcsname{\let\PY@bf=\textbf\def\PY@tc##1{\textcolor[rgb]{0.00,0.50,0.00}{##1}}}
\expandafter\def\csname PY@tok@kp\endcsname{\def\PY@tc##1{\textcolor[rgb]{0.00,0.50,0.00}{##1}}}
\expandafter\def\csname PY@tok@kt\endcsname{\def\PY@tc##1{\textcolor[rgb]{0.69,0.00,0.25}{##1}}}
\expandafter\def\csname PY@tok@o\endcsname{\def\PY@tc##1{\textcolor[rgb]{0.40,0.40,0.40}{##1}}}
\expandafter\def\csname PY@tok@ow\endcsname{\let\PY@bf=\textbf\def\PY@tc##1{\textcolor[rgb]{0.67,0.13,1.00}{##1}}}
\expandafter\def\csname PY@tok@nb\endcsname{\def\PY@tc##1{\textcolor[rgb]{0.00,0.50,0.00}{##1}}}
\expandafter\def\csname PY@tok@nf\endcsname{\def\PY@tc##1{\textcolor[rgb]{0.00,0.00,1.00}{##1}}}
\expandafter\def\csname PY@tok@nc\endcsname{\let\PY@bf=\textbf\def\PY@tc##1{\textcolor[rgb]{0.00,0.00,1.00}{##1}}}
\expandafter\def\csname PY@tok@nn\endcsname{\let\PY@bf=\textbf\def\PY@tc##1{\textcolor[rgb]{0.00,0.00,1.00}{##1}}}
\expandafter\def\csname PY@tok@ne\endcsname{\let\PY@bf=\textbf\def\PY@tc##1{\textcolor[rgb]{0.82,0.25,0.23}{##1}}}
\expandafter\def\csname PY@tok@nv\endcsname{\def\PY@tc##1{\textcolor[rgb]{0.10,0.09,0.49}{##1}}}
\expandafter\def\csname PY@tok@no\endcsname{\def\PY@tc##1{\textcolor[rgb]{0.53,0.00,0.00}{##1}}}
\expandafter\def\csname PY@tok@nl\endcsname{\def\PY@tc##1{\textcolor[rgb]{0.63,0.63,0.00}{##1}}}
\expandafter\def\csname PY@tok@ni\endcsname{\let\PY@bf=\textbf\def\PY@tc##1{\textcolor[rgb]{0.60,0.60,0.60}{##1}}}
\expandafter\def\csname PY@tok@na\endcsname{\def\PY@tc##1{\textcolor[rgb]{0.49,0.56,0.16}{##1}}}
\expandafter\def\csname PY@tok@nt\endcsname{\let\PY@bf=\textbf\def\PY@tc##1{\textcolor[rgb]{0.00,0.50,0.00}{##1}}}
\expandafter\def\csname PY@tok@nd\endcsname{\def\PY@tc##1{\textcolor[rgb]{0.67,0.13,1.00}{##1}}}
\expandafter\def\csname PY@tok@s\endcsname{\def\PY@tc##1{\textcolor[rgb]{0.73,0.13,0.13}{##1}}}
\expandafter\def\csname PY@tok@sd\endcsname{\let\PY@it=\textit\def\PY@tc##1{\textcolor[rgb]{0.73,0.13,0.13}{##1}}}
\expandafter\def\csname PY@tok@si\endcsname{\let\PY@bf=\textbf\def\PY@tc##1{\textcolor[rgb]{0.73,0.40,0.53}{##1}}}
\expandafter\def\csname PY@tok@se\endcsname{\let\PY@bf=\textbf\def\PY@tc##1{\textcolor[rgb]{0.73,0.40,0.13}{##1}}}
\expandafter\def\csname PY@tok@sr\endcsname{\def\PY@tc##1{\textcolor[rgb]{0.73,0.40,0.53}{##1}}}
\expandafter\def\csname PY@tok@ss\endcsname{\def\PY@tc##1{\textcolor[rgb]{0.10,0.09,0.49}{##1}}}
\expandafter\def\csname PY@tok@sx\endcsname{\def\PY@tc##1{\textcolor[rgb]{0.00,0.50,0.00}{##1}}}
\expandafter\def\csname PY@tok@m\endcsname{\def\PY@tc##1{\textcolor[rgb]{0.40,0.40,0.40}{##1}}}
\expandafter\def\csname PY@tok@gh\endcsname{\let\PY@bf=\textbf\def\PY@tc##1{\textcolor[rgb]{0.00,0.00,0.50}{##1}}}
\expandafter\def\csname PY@tok@gu\endcsname{\let\PY@bf=\textbf\def\PY@tc##1{\textcolor[rgb]{0.50,0.00,0.50}{##1}}}
\expandafter\def\csname PY@tok@gd\endcsname{\def\PY@tc##1{\textcolor[rgb]{0.63,0.00,0.00}{##1}}}
\expandafter\def\csname PY@tok@gi\endcsname{\def\PY@tc##1{\textcolor[rgb]{0.00,0.63,0.00}{##1}}}
\expandafter\def\csname PY@tok@gr\endcsname{\def\PY@tc##1{\textcolor[rgb]{1.00,0.00,0.00}{##1}}}
\expandafter\def\csname PY@tok@ge\endcsname{\let\PY@it=\textit}
\expandafter\def\csname PY@tok@gs\endcsname{\let\PY@bf=\textbf}
\expandafter\def\csname PY@tok@gp\endcsname{\let\PY@bf=\textbf\def\PY@tc##1{\textcolor[rgb]{0.00,0.00,0.50}{##1}}}
\expandafter\def\csname PY@tok@go\endcsname{\def\PY@tc##1{\textcolor[rgb]{0.53,0.53,0.53}{##1}}}
\expandafter\def\csname PY@tok@gt\endcsname{\def\PY@tc##1{\textcolor[rgb]{0.00,0.27,0.87}{##1}}}
\expandafter\def\csname PY@tok@err\endcsname{\def\PY@bc##1{\setlength{\fboxsep}{0pt}\fcolorbox[rgb]{1.00,0.00,0.00}{1,1,1}{\strut ##1}}}
\expandafter\def\csname PY@tok@kc\endcsname{\let\PY@bf=\textbf\def\PY@tc##1{\textcolor[rgb]{0.00,0.50,0.00}{##1}}}
\expandafter\def\csname PY@tok@kd\endcsname{\let\PY@bf=\textbf\def\PY@tc##1{\textcolor[rgb]{0.00,0.50,0.00}{##1}}}
\expandafter\def\csname PY@tok@kn\endcsname{\let\PY@bf=\textbf\def\PY@tc##1{\textcolor[rgb]{0.00,0.50,0.00}{##1}}}
\expandafter\def\csname PY@tok@kr\endcsname{\let\PY@bf=\textbf\def\PY@tc##1{\textcolor[rgb]{0.00,0.50,0.00}{##1}}}
\expandafter\def\csname PY@tok@bp\endcsname{\def\PY@tc##1{\textcolor[rgb]{0.00,0.50,0.00}{##1}}}
\expandafter\def\csname PY@tok@fm\endcsname{\def\PY@tc##1{\textcolor[rgb]{0.00,0.00,1.00}{##1}}}
\expandafter\def\csname PY@tok@vc\endcsname{\def\PY@tc##1{\textcolor[rgb]{0.10,0.09,0.49}{##1}}}
\expandafter\def\csname PY@tok@vg\endcsname{\def\PY@tc##1{\textcolor[rgb]{0.10,0.09,0.49}{##1}}}
\expandafter\def\csname PY@tok@vi\endcsname{\def\PY@tc##1{\textcolor[rgb]{0.10,0.09,0.49}{##1}}}
\expandafter\def\csname PY@tok@vm\endcsname{\def\PY@tc##1{\textcolor[rgb]{0.10,0.09,0.49}{##1}}}
\expandafter\def\csname PY@tok@sa\endcsname{\def\PY@tc##1{\textcolor[rgb]{0.73,0.13,0.13}{##1}}}
\expandafter\def\csname PY@tok@sb\endcsname{\def\PY@tc##1{\textcolor[rgb]{0.73,0.13,0.13}{##1}}}
\expandafter\def\csname PY@tok@sc\endcsname{\def\PY@tc##1{\textcolor[rgb]{0.73,0.13,0.13}{##1}}}
\expandafter\def\csname PY@tok@dl\endcsname{\def\PY@tc##1{\textcolor[rgb]{0.73,0.13,0.13}{##1}}}
\expandafter\def\csname PY@tok@s2\endcsname{\def\PY@tc##1{\textcolor[rgb]{0.73,0.13,0.13}{##1}}}
\expandafter\def\csname PY@tok@sh\endcsname{\def\PY@tc##1{\textcolor[rgb]{0.73,0.13,0.13}{##1}}}
\expandafter\def\csname PY@tok@s1\endcsname{\def\PY@tc##1{\textcolor[rgb]{0.73,0.13,0.13}{##1}}}
\expandafter\def\csname PY@tok@mb\endcsname{\def\PY@tc##1{\textcolor[rgb]{0.40,0.40,0.40}{##1}}}
\expandafter\def\csname PY@tok@mf\endcsname{\def\PY@tc##1{\textcolor[rgb]{0.40,0.40,0.40}{##1}}}
\expandafter\def\csname PY@tok@mh\endcsname{\def\PY@tc##1{\textcolor[rgb]{0.40,0.40,0.40}{##1}}}
\expandafter\def\csname PY@tok@mi\endcsname{\def\PY@tc##1{\textcolor[rgb]{0.40,0.40,0.40}{##1}}}
\expandafter\def\csname PY@tok@il\endcsname{\def\PY@tc##1{\textcolor[rgb]{0.40,0.40,0.40}{##1}}}
\expandafter\def\csname PY@tok@mo\endcsname{\def\PY@tc##1{\textcolor[rgb]{0.40,0.40,0.40}{##1}}}
\expandafter\def\csname PY@tok@ch\endcsname{\let\PY@it=\textit\def\PY@tc##1{\textcolor[rgb]{0.25,0.50,0.50}{##1}}}
\expandafter\def\csname PY@tok@cm\endcsname{\let\PY@it=\textit\def\PY@tc##1{\textcolor[rgb]{0.25,0.50,0.50}{##1}}}
\expandafter\def\csname PY@tok@cpf\endcsname{\let\PY@it=\textit\def\PY@tc##1{\textcolor[rgb]{0.25,0.50,0.50}{##1}}}
\expandafter\def\csname PY@tok@c1\endcsname{\let\PY@it=\textit\def\PY@tc##1{\textcolor[rgb]{0.25,0.50,0.50}{##1}}}
\expandafter\def\csname PY@tok@cs\endcsname{\let\PY@it=\textit\def\PY@tc##1{\textcolor[rgb]{0.25,0.50,0.50}{##1}}}

\def\PYZbs{\char`\\}
\def\PYZus{\char`\_}
\def\PYZob{\char`\{}
\def\PYZcb{\char`\}}
\def\PYZca{\char`\^}
\def\PYZam{\char`\&}
\def\PYZlt{\char`\<}
\def\PYZgt{\char`\>}
\def\PYZsh{\char`\#}
\def\PYZpc{\char`\%}
\def\PYZdl{\char`\$}
\def\PYZhy{\char`\-}
\def\PYZsq{\char`\'}
\def\PYZdq{\char`\"}
\def\PYZti{\char`\~}
% for compatibility with earlier versions
\def\PYZat{@}
\def\PYZlb{[}
\def\PYZrb{]}
\makeatother


    % Exact colors from NB
    \definecolor{incolor}{rgb}{0.0, 0.0, 0.5}
    \definecolor{outcolor}{rgb}{0.545, 0.0, 0.0}



    
    % Prevent overflowing lines due to hard-to-break entities
    \sloppy 
    % Setup hyperref package
    \hypersetup{
      breaklinks=true,  % so long urls are correctly broken across lines
      colorlinks=true,
      urlcolor=urlcolor,
      linkcolor=linkcolor,
      citecolor=citecolor,
      }
    % Slightly bigger margins than the latex defaults
    
    \geometry{verbose,tmargin=1in,bmargin=1in,lmargin=1in,rmargin=1in}
    
    

    \begin{document}
    
    
    \maketitle
    
    

    
    \hypertarget{ux306fux3058ux3081ux306b}{%
\section{はじめに}\label{ux306fux3058ux3081ux306b}}

    \hypertarget{ux66f8ux3053ux3046ux3068ux601dux3063ux305fux304dux3063ux304bux3051}{%
\subsection{書こうと思ったきっかけ}\label{ux66f8ux3053ux3046ux3068ux601dux3063ux305fux304dux3063ux304bux3051}}

    実習中論文を頂いて読む機会が多いが、医学論文は薬は新しい治療方法の成績を比較しているのが多い。
だいたいKaplan-Meier曲線が使われているので(偏見)、論文を読むときにはもちろん、将来自分が論文を執筆するときに、何が行われているのか少しでも理解できていればいいかなと思いまとめてみた。

    \hypertarget{kaplan-meier-ux66f2ux7ddaux3068ux306f}{%
\subsection{Kaplan-Meier
曲線とは}\label{kaplan-meier-ux66f2ux7ddaux3068ux306f}}

    \href{https://ja.wikipedia.org/wiki/\%E7\%94\%9F\%E5\%AD\%98\%E7\%8E\%87\%E6\%9B\%B2\%E7\%B7\%9A}{wikipedia}によると「治療を行った後の患者の生存率をグラフにしたもの」ということらしい。
ただ、実際には問題がいくつかあって

\begin{itemize}
\tightlist
\item
  観察期間は限られている(患者さんを何年もフォローすることはできない)
\item
  観察不能になることがある(病院移った、他の要因で死亡したなど)
\item
  患者さんによって観察開始時間は様々(みんな一緒に罹患したり治療を開始するわけではない)
\end{itemize}

そういった問題をうまく対処してくれるのがKaplan-Meier曲線。

    Kaplan-Meier曲線の縦軸は累積生存率で、 \[
  S(t) = \prod_{j}^t (1-\frac{d_j}{n_j})
\] で表されます。
ここで\(d_j\)は時間jにおける死亡者数で\(n_j\)は時間jにおける生存者数です。

pythonのコードでKaplan-Meier曲線を書いていこうと思います。
pythonのライブラリにlifelinesという便利なライブラリがあるのでそれを使っていこうと思います。

\texttt{pip\ install\ lifelines}

    \hypertarget{ux30c7ux30fcux30bfux306eux6e96ux5099}{%
\section{データの準備}\label{ux30c7ux30fcux30bfux306eux6e96ux5099}}

    ある病気に対する治療法の予後のKaplan-Meier曲線を描きたいとします。
ある期間内に観察された患者の経過をみることになると思いますが、患者さんによって発症する時間も様々なので、イメージとして下の図のようになると思います。

    \begin{Verbatim}[commandchars=\\\{\}]
{\color{incolor}In [{\color{incolor}256}]:} \PY{k+kn}{import} \PY{n+nn}{pandas} \PY{k}{as} \PY{n+nn}{pd}
          \PY{k+kn}{import} \PY{n+nn}{numpy} \PY{k}{as} \PY{n+nn}{np}
          \PY{k+kn}{import} \PY{n+nn}{matplotlib}\PY{n+nn}{.}\PY{n+nn}{pyplot} \PY{k}{as} \PY{n+nn}{plt}
          \PY{k+kn}{from} \PY{n+nn}{lifelines} \PY{k}{import} \PY{n}{KaplanMeierFitter}
          \PY{k+kn}{from} \PY{n+nn}{lifelines}\PY{n+nn}{.}\PY{n+nn}{plotting} \PY{k}{import} \PY{n}{plot\PYZus{}lifetimes}
          \PY{k+kn}{from} \PY{n+nn}{numpy}\PY{n+nn}{.}\PY{n+nn}{random} \PY{k}{import} \PY{n}{uniform}\PY{p}{,} \PY{n}{exponential}
          \PY{o}{\PYZpc{}}\PY{k}{matplotlib} inline
          \PY{o}{\PYZpc{}}\PY{k}{config} InlineBackend.figure\PYZus{}format = \PYZsq{}retina\PYZsq{}
          
          \PY{n}{N} \PY{o}{=} \PY{l+m+mi}{50}
          \PY{n}{current\PYZus{}time} \PY{o}{=} \PY{l+m+mi}{48}
          \PY{n}{birth} \PY{o}{=} \PY{n}{np}\PY{o}{.}\PY{n}{random}\PY{o}{.}\PY{n}{randint}\PY{p}{(}\PY{l+m+mi}{0}\PY{p}{,}\PY{l+m+mi}{45}\PY{p}{,}\PY{l+m+mi}{50}\PY{p}{)}
          \PY{n}{actual\PYZus{}lifetimes} \PY{o}{=} \PY{n}{np}\PY{o}{.}\PY{n}{array}\PY{p}{(}\PY{p}{[}\PY{p}{[}\PY{n+nb}{int}\PY{p}{(}\PY{n}{exponential}\PY{p}{(}\PY{l+m+mi}{20}\PY{p}{)}\PY{p}{)}\PY{o}{+}\PY{l+m+mi}{1}\PY{p}{,} \PY{n+nb}{int}\PY{p}{(}\PY{n}{exponential}\PY{p}{(}\PY{l+m+mi}{10}\PY{p}{)}\PY{p}{)}\PY{o}{+}\PY{l+m+mi}{1}\PY{p}{]}\PY{p}{[}\PY{n}{uniform}\PY{p}{(}\PY{p}{)} \PY{o}{\PYZlt{}} \PY{l+m+mf}{0.5}\PY{p}{]} \PY{k}{for} \PY{n}{i} \PY{o+ow}{in} \PY{n+nb}{range}\PY{p}{(}\PY{n}{N}\PY{p}{)}\PY{p}{]}\PY{p}{)}
          \PY{n}{observed\PYZus{}lifetimes} \PY{o}{=} \PY{n}{np}\PY{o}{.}\PY{n}{minimum}\PY{p}{(}\PY{p}{(}\PY{n}{actual\PYZus{}lifetimes}\PY{o}{+}\PY{n}{birth}\PY{p}{)}\PY{p}{,} \PY{n}{current\PYZus{}time}\PY{p}{)}
          \PY{n}{observed} \PY{o}{=} \PY{p}{(}\PY{n}{birth} \PY{o}{+} \PY{n}{actual\PYZus{}lifetimes}\PY{p}{)} \PY{o}{\PYZlt{}} \PY{n}{current\PYZus{}time}
          
          \PY{n}{plt}\PY{o}{.}\PY{n}{xlim}\PY{p}{(}\PY{l+m+mi}{0}\PY{p}{,} \PY{l+m+mi}{50}\PY{p}{)}
          \PY{n}{plt}\PY{o}{.}\PY{n}{vlines}\PY{p}{(}\PY{l+m+mi}{48}\PY{p}{,} \PY{l+m+mi}{0}\PY{p}{,} \PY{l+m+mi}{50}\PY{p}{,} \PY{n}{lw}\PY{o}{=}\PY{l+m+mi}{2}\PY{p}{,} \PY{n}{linestyles}\PY{o}{=}\PY{l+s+s1}{\PYZsq{}}\PY{l+s+s1}{\PYZhy{}\PYZhy{}}\PY{l+s+s1}{\PYZsq{}}\PY{p}{,}\PY{n}{colors}\PY{o}{=}\PY{l+s+s1}{\PYZsq{}}\PY{l+s+s1}{red}\PY{l+s+s1}{\PYZsq{}}\PY{p}{)}
          \PY{n}{plt}\PY{o}{.}\PY{n}{xlabel}\PY{p}{(}\PY{l+s+s2}{\PYZdq{}}\PY{l+s+s2}{time}\PY{l+s+s2}{\PYZdq{}}\PY{p}{)}
          \PY{n}{plt}\PY{o}{.}\PY{n}{title}\PY{p}{(}\PY{l+s+s2}{\PYZdq{}}\PY{l+s+s2}{Births and deaths of our population, at \PYZdl{}t=48\PYZdl{}}\PY{l+s+s2}{\PYZdq{}}\PY{p}{)}
          \PY{n}{plot\PYZus{}lifetimes}\PY{p}{(}\PY{n}{observed\PYZus{}lifetimes}\PY{o}{\PYZhy{}}\PY{n}{birth}\PY{p}{,} \PY{n}{event\PYZus{}observed}\PY{o}{=}\PY{n}{observed}\PY{p}{,}\PY{n}{birthtimes}\PY{o}{=}\PY{n}{birth}\PY{p}{)}
          \PY{n+nb}{print}\PY{p}{(}\PY{l+s+s2}{\PYZdq{}}\PY{l+s+s2}{Observed lifetimes at time }\PY{l+s+si}{\PYZpc{}d}\PY{l+s+s2}{:}\PY{l+s+se}{\PYZbs{}n}\PY{l+s+s2}{\PYZdq{}} \PY{o}{\PYZpc{}} \PY{p}{(}\PY{n}{current\PYZus{}time}\PY{p}{)}\PY{p}{,} \PY{n}{observed\PYZus{}lifetimes}\PY{p}{)}
\end{Verbatim}


    \begin{center}
    \adjustimage{max size={0.9\linewidth}{0.9\paperheight}}{output_8_0.png}
    \end{center}
    { \hspace*{\fill} \\}
    
    \begin{Verbatim}[commandchars=\\\{\}]
Observed lifetimes at time 48:
 [48 47 27 48 48 40 34  3 46 41 40 48 30 48 40 46 48 34 48 48 15 36 38 31
 46 22 44 16 42 48 45 19 46 44 31 16 33  9 32 42 11 36 48 48 44 48 14 39
 37 45]

    \end{Verbatim}

    赤線では期間内(t=48)に患者さんが死亡したケース、青線は観測不能になったケースです(観察期間終わったからそれ以上追えない)。

簡単のため、観測不能になったケースは観測期間が終わった場合のみとします。観察期間の長さでソートすると、下の図のようになります。

    \begin{Verbatim}[commandchars=\\\{\}]
{\color{incolor}In [{\color{incolor}237}]:} \PY{n}{s} \PY{o}{=} \PY{n}{pd}\PY{o}{.}\PY{n}{Series}\PY{p}{(}\PY{n}{observed\PYZus{}lifetimes}\PY{o}{\PYZhy{}}\PY{n}{birth}\PY{p}{,}\PY{n}{observed}\PY{p}{)}
          \PY{n}{s} \PY{o}{=} \PY{n}{s}\PY{o}{.}\PY{n}{sort\PYZus{}values}\PY{p}{(}\PY{p}{)}
          \PY{n}{s}\PY{o}{.}\PY{n}{name} \PY{o}{=} \PY{l+s+s1}{\PYZsq{}}\PY{l+s+s1}{time}\PY{l+s+s1}{\PYZsq{}}
          \PY{n}{observed}\PY{p}{,} \PY{n}{result\PYZus{}lifetimes} \PY{o}{=} \PY{n}{s}\PY{o}{.}\PY{n}{index}\PY{p}{,} \PY{n}{s}\PY{o}{.}\PY{n}{values}
          
          \PY{n}{plt}\PY{o}{.}\PY{n}{xlim}\PY{p}{(}\PY{l+m+mi}{0}\PY{p}{,}\PY{l+m+mi}{50}\PY{p}{)}
          \PY{c+c1}{\PYZsh{}plt.vlines(22, 0, 30, lw=2, linestyles=\PYZsq{}\PYZhy{}\PYZhy{}\PYZsq{},colors=\PYZsq{}red\PYZsq{})}
          \PY{n}{plt}\PY{o}{.}\PY{n}{xlabel}\PY{p}{(}\PY{l+s+s2}{\PYZdq{}}\PY{l+s+s2}{time}\PY{l+s+s2}{\PYZdq{}}\PY{p}{)}
          \PY{n}{plt}\PY{o}{.}\PY{n}{title}\PY{p}{(}\PY{l+s+s2}{\PYZdq{}}\PY{l+s+s2}{Births and deaths of our population, at \PYZdl{}t=22\PYZdl{}}\PY{l+s+s2}{\PYZdq{}}\PY{p}{)}
          \PY{n}{plot\PYZus{}lifetimes}\PY{p}{(}\PY{n}{result\PYZus{}lifetimes}\PY{p}{,} \PY{n}{event\PYZus{}observed}\PY{o}{=}\PY{n}{observed}\PY{p}{)}
\end{Verbatim}


    \begin{center}
    \adjustimage{max size={0.9\linewidth}{0.9\paperheight}}{output_10_0.png}
    \end{center}
    { \hspace*{\fill} \\}
    
    \hypertarget{ux5b9fux969bux306eux8a08ux7b97}{%
\section{実際の計算}\label{ux5b9fux969bux306eux8a08ux7b97}}

    さっき作ったデータを使って、累積生存率を計算していきます。たとえばt=5における累積生存率を計算するとき、

    \begin{Verbatim}[commandchars=\\\{\}]
{\color{incolor}In [{\color{incolor}238}]:} \PY{n}{pd}\PY{o}{.}\PY{n}{DataFrame}\PY{p}{(}\PY{n}{s}\PY{p}{)}\PY{o}{.}\PY{n}{head}\PY{p}{(}\PY{l+m+mi}{12}\PY{p}{)}
\end{Verbatim}


\begin{Verbatim}[commandchars=\\\{\}]
{\color{outcolor}Out[{\color{outcolor}238}]:}        time
          False     1
          True      1
          True      1
          True      3
          True      4
          False     4
          False     4
          True      4
          False     5
          False     5
          True      5
          False     5
\end{Verbatim}
            
    Trueが患者さんが死亡したのを観測できた事例、Falseが観察期間終了により観測できなかった事例です。観察期間の短い順に見ていくと、\\
t=1のとき、50人のうち2人が死亡,1人が打ち切り。\\
t=2のとき、死亡者、打ち切りなし。 t=3の時は47人のうち1人が死亡。\\
t=4の時は46人のうち2人が死亡、2人が打ち切り。\\
t=5の時は42人中1人死亡、3人が打ち切り。 よって\(S(5)\)は   \[
  S(5) = (1-\frac{1}{50})(1-\frac{0}{47})(1-\frac{1}{47})(1-\frac{2}{46})(1-\frac{1}{42})
       = 0.8956028368794325
\] となります。

    \hypertarget{ux7d50ux679cux306eux30d7ux30edux30c3ux30c8}{%
\section{結果のプロット}\label{ux7d50ux679cux306eux30d7ux30edux30c3ux30c8}}

    ここら辺の計算と描画はライブラリに任せます。\\
やってることは各期間の点における累積生存率をプロットして線を横に伸ばすだけです。

    \begin{Verbatim}[commandchars=\\\{\}]
{\color{incolor}In [{\color{incolor}233}]:} \PY{n}{kmf} \PY{o}{=} \PY{n}{KaplanMeierFitter}\PY{p}{(}\PY{p}{)}
          \PY{n}{kmf}\PY{o}{.}\PY{n}{fit}\PY{p}{(}\PY{n}{result\PYZus{}lifetimes}\PY{p}{,}\PY{n}{event\PYZus{}observed}\PY{o}{=}\PY{n}{observed}\PY{p}{)}
          \PY{n}{kmf}\PY{o}{.}\PY{n}{plot}\PY{p}{(}\PY{p}{)}
          \PY{n}{plt}\PY{o}{.}\PY{n}{title}\PY{p}{(}\PY{l+s+s1}{\PYZsq{}}\PY{l+s+s1}{Survival function of political regimes}\PY{l+s+s1}{\PYZsq{}}\PY{p}{)}\PY{p}{;}
\end{Verbatim}


    \begin{center}
    \adjustimage{max size={0.9\linewidth}{0.9\paperheight}}{output_17_0.png}
    \end{center}
    { \hspace*{\fill} \\}
    
    \hypertarget{ux4e73ux304cux3093ux30c7ux30fcux30bfux30bbux30c3ux30c8ux3092ux7528ux3044ux3066ux89e3ux6790ux3092ux884cux3063ux305fux4f8b}{%
\section{乳がんデータセットを用いて解析を行った例}\label{ux4e73ux304cux3093ux30c7ux30fcux30bfux30bbux30c3ux30c8ux3092ux7528ux3044ux3066ux89e3ux6790ux3092ux884cux3063ux305fux4f8b}}

    一応これで生存曲線は描けた訳ですが、もし治療の有効性を示すのであれば、他の治療群と比較して、有効性を示さなければいけません。\\
ここから使われる統計手法はLog-rank検定、一般化wilcoxon検定など様々ですが、ここではcox比例ハザードを使った回帰分析を例にして解析を進めていきます。\\
cox比例ハザードモデルに関する説明は\href{http://www012.upp.so-net.ne.jp/doi/biostat/CT39/Cox.pdf}{ここ}がわかりやすかった。\\
lifelinesライブラリに乳がんの予後のデータセットがあるのでそれを用います。\\
乳がんの治療でどのような因子が予後のに影響を及ぼすのか調べてみます。

    \begin{Verbatim}[commandchars=\\\{\}]
{\color{incolor}In [{\color{incolor}263}]:} \PY{c+c1}{\PYZsh{} hormone therapyの有無で2群に分けKaplan\PYZhy{}Meier曲線を描いた例}
          \PY{k+kn}{from} \PY{n+nn}{lifelines}\PY{n+nn}{.}\PY{n+nn}{datasets} \PY{k}{import} \PY{n}{load\PYZus{}gbsg2}
          \PY{n}{df} \PY{o}{=} \PY{n}{load\PYZus{}gbsg2}\PY{p}{(}\PY{p}{)}
          \PY{n}{ax} \PY{o}{=} \PY{k+kc}{None}
          \PY{k}{for} \PY{n}{name}\PY{p}{,} \PY{n}{group} \PY{o+ow}{in} \PY{n}{df}\PY{o}{.}\PY{n}{groupby}\PY{p}{(}\PY{l+s+s1}{\PYZsq{}}\PY{l+s+s1}{horTh}\PY{l+s+s1}{\PYZsq{}}\PY{p}{)}\PY{p}{:}
              \PY{n}{kmf} \PY{o}{=} \PY{n}{KaplanMeierFitter}\PY{p}{(}\PY{p}{)}
              \PY{n}{kmf}\PY{o}{.}\PY{n}{fit}\PY{p}{(}\PY{n}{group}\PY{p}{[}\PY{l+s+s1}{\PYZsq{}}\PY{l+s+s1}{time}\PY{l+s+s1}{\PYZsq{}}\PY{p}{]}\PY{p}{,} \PY{n}{event\PYZus{}observed}\PY{o}{=}\PY{n}{group}\PY{p}{[}\PY{l+s+s1}{\PYZsq{}}\PY{l+s+s1}{cens}\PY{l+s+s1}{\PYZsq{}}\PY{p}{]}\PY{p}{,}
                      \PY{n}{label} \PY{o}{=} \PY{l+s+s1}{\PYZsq{}}\PY{l+s+s1}{hormone therapy =}\PY{l+s+s1}{\PYZsq{}} \PY{o}{+} \PY{n+nb}{str}\PY{p}{(}\PY{n}{name}\PY{p}{)}\PY{p}{)}
              \PY{c+c1}{\PYZsh{} 描画する Axes を指定。None を渡すとエラーになるので場合分け}
              \PY{k}{if} \PY{n}{ax} \PY{o+ow}{is} \PY{k+kc}{None}\PY{p}{:}
                  \PY{n}{ax} \PY{o}{=} \PY{n}{kmf}\PY{o}{.}\PY{n}{plot}\PY{p}{(}\PY{p}{)}
              \PY{k}{else}\PY{p}{:}
                  \PY{n}{ax} \PY{o}{=} \PY{n}{kmf}\PY{o}{.}\PY{n}{plot}\PY{p}{(}\PY{n}{ax}\PY{o}{=}\PY{n}{ax}\PY{p}{)}
          \PY{n}{plt}\PY{o}{.}\PY{n}{title}\PY{p}{(}\PY{l+s+s1}{\PYZsq{}}\PY{l+s+s1}{Kaplan\PYZhy{}Meier Curve}\PY{l+s+s1}{\PYZsq{}}\PY{p}{)}
          \PY{n}{plt}\PY{o}{.}\PY{n}{show}\PY{p}{(}\PY{p}{)}
\end{Verbatim}


    \begin{center}
    \adjustimage{max size={0.9\linewidth}{0.9\paperheight}}{output_20_0.png}
    \end{center}
    { \hspace*{\fill} \\}
    
    \begin{Verbatim}[commandchars=\\\{\}]
{\color{incolor}In [{\color{incolor}264}]:} \PY{k+kn}{from} \PY{n+nn}{lifelines} \PY{k}{import} \PY{n}{CoxPHFitter}
          \PY{n}{cph} \PY{o}{=} \PY{n}{CoxPHFitter}\PY{p}{(}\PY{p}{)}
          \PY{n}{df} \PY{o}{=} \PY{n}{pd}\PY{o}{.}\PY{n}{get\PYZus{}dummies}\PY{p}{(}\PY{n}{df}\PY{p}{,}\PY{n}{columns}\PY{o}{=}\PY{p}{[}\PY{l+s+s1}{\PYZsq{}}\PY{l+s+s1}{horTh}\PY{l+s+s1}{\PYZsq{}}\PY{p}{,}\PY{l+s+s1}{\PYZsq{}}\PY{l+s+s1}{menostat}\PY{l+s+s1}{\PYZsq{}}\PY{p}{]}\PY{p}{,}\PY{n}{drop\PYZus{}first}\PY{o}{=}\PY{k+kc}{True}\PY{p}{)}
          \PY{n}{df}\PY{o}{.}\PY{n}{drop}\PY{p}{(}\PY{l+s+s1}{\PYZsq{}}\PY{l+s+s1}{tgrade}\PY{l+s+s1}{\PYZsq{}}\PY{p}{,}\PY{n}{axis} \PY{o}{=} \PY{l+m+mi}{1}\PY{p}{,}\PY{n}{inplace} \PY{o}{=} \PY{k+kc}{True}\PY{p}{)}
          \PY{n}{cph}\PY{o}{.}\PY{n}{fit}\PY{p}{(}\PY{n}{df}\PY{p}{,} \PY{n}{duration\PYZus{}col}\PY{o}{=}\PY{l+s+s1}{\PYZsq{}}\PY{l+s+s1}{time}\PY{l+s+s1}{\PYZsq{}}\PY{p}{,} \PY{n}{event\PYZus{}col}\PY{o}{=}\PY{l+s+s1}{\PYZsq{}}\PY{l+s+s1}{cens}\PY{l+s+s1}{\PYZsq{}}\PY{p}{,} \PY{n}{show\PYZus{}progress}\PY{o}{=}\PY{k+kc}{True}\PY{p}{)}
\end{Verbatim}


    \begin{Verbatim}[commandchars=\\\{\}]
Iteration 1: norm\_delta = 0.63685, step\_size = 0.95000, ll = -1788.10474, seconds\_since\_start = 0.0
Iteration 2: norm\_delta = 0.28516, step\_size = 0.95000, ll = -1781.26562, seconds\_since\_start = 0.0
Iteration 3: norm\_delta = 0.09501, step\_size = 0.95000, ll = -1745.22376, seconds\_since\_start = 0.1
Iteration 4: norm\_delta = 0.02423, step\_size = 0.95000, ll = -1740.89966, seconds\_since\_start = 0.1
Iteration 5: norm\_delta = 0.00263, step\_size = 0.95000, ll = -1740.66192, seconds\_since\_start = 0.1
Iteration 6: norm\_delta = 0.00015, step\_size = 0.95000, ll = -1740.65941, seconds\_since\_start = 0.1
Iteration 7: norm\_delta = 0.00001, step\_size = 0.95000, ll = -1740.65940, seconds\_since\_start = 0.1
Convergence completed after 7 iterations.

    \end{Verbatim}

\begin{Verbatim}[commandchars=\\\{\}]
{\color{outcolor}Out[{\color{outcolor}264}]:} <lifelines.CoxPHFitter: fitted with 686 observations, 387 censored>
\end{Verbatim}
            
    \begin{Verbatim}[commandchars=\\\{\}]
{\color{incolor}In [{\color{incolor}265}]:} \PY{n}{cph}\PY{o}{.}\PY{n}{print\PYZus{}summary}\PY{p}{(}\PY{p}{)}
\end{Verbatim}


    \begin{Verbatim}[commandchars=\\\{\}]
n=686, number of events=299

                coef  exp(coef)  se(coef)       z      p  lower 0.95  upper 0.95     
age          -0.0105     0.9896    0.0093 -1.1296 0.2587     -0.0287      0.0077     
tsize         0.0084     1.0084    0.0039  2.1173 0.0342      0.0006      0.0161    *
pnodes        0.0498     1.0511    0.0074  6.7329 0.0000      0.0353      0.0643  ***
progrec      -0.0026     0.9974    0.0006 -4.4522 0.0000     -0.0037     -0.0015  ***
estrec        0.0002     1.0002    0.0005  0.3843 0.7008     -0.0007      0.0011     
horTh\_yes    -0.3643     0.6947    0.1284 -2.8371 0.0046     -0.6159     -0.1126   **
menostat\_Pre -0.2767     0.7583    0.1822 -1.5191 0.1287     -0.6338      0.0803     
---
Signif. codes:  0 '***' 0.001 '**' 0.01 '*' 0.05 '.' 0.1 ' ' 1 

Concordance = 0.687
Likelihood ratio test = 94.891 on 7 df, p=0.00000

    \end{Verbatim}

    \begin{Verbatim}[commandchars=\\\{\}]
{\color{incolor}In [{\color{incolor}266}]:} \PY{n}{cph}\PY{o}{.}\PY{n}{plot}\PY{p}{(}\PY{p}{)}
\end{Verbatim}


\begin{Verbatim}[commandchars=\\\{\}]
{\color{outcolor}Out[{\color{outcolor}266}]:} <matplotlib.axes.\_subplots.AxesSubplot at 0x1a20f189e8>
\end{Verbatim}
            
    \begin{center}
    \adjustimage{max size={0.9\linewidth}{0.9\paperheight}}{output_23_1.png}
    \end{center}
    { \hspace*{\fill} \\}
    
    \hypertarget{ux307eux3068ux3081}{%
\section{まとめ}\label{ux307eux3068ux3081}}

    Kaplan-Meier曲線がどのように作られているか分かった。\\
pythonを用いて割と簡単に統計解析ができた。\\
統計モデルについても定義からもう少し詳しく説明できたらよかったが、latex使うのめんどくさくて諦めた。リンク参考にしてください。

    \hypertarget{ux53c2ux8003ux6587ux732e}{%
\subsubsection{参考文献}\label{ux53c2ux8003ux6587ux732e}}

https://lifelines.readthedocs.io/en/latest/index.html\\
https://istat.co.jp/sk\_commentary/kaplan\_meier\\
http://www.emalliance.org/\%E6\%9C\%AA\%E5\%88\%86\%E9\%A1\%9E/\%E7\%AC\%AC\%EF\%BC\%97\%E5\%9B\%9Eema-jc\%E3\%80\%80\%E8\%A7\%A3\%E8\%AA\%AC\%E3\%80\%80\%E3\%82\%AB\%E3\%83\%97\%E3\%83\%A9\%E3\%83\%B3\%E3\%83\%9E\%E3\%82\%A4\%E3\%83\%A4\%E3\%83\%BC\%E7\%94\%9F\%E5\%AD\%98\%E6\%9B\%B2\%E7\%B7\%9A


    % Add a bibliography block to the postdoc
    
    
    
    \end{document}
